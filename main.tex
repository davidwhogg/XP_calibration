\documentclass{article}
\usepackage[utf8]{inputenc}
\usepackage[a4paper]{geometry}
\usepackage{setspace}
\usepackage{amsmath, amsfonts, mathrsfs}

% typesetting issues
\setstretch{1.08}
\addtolength{\topmargin}{-10mm}
\addtolength{\textheight}{40mm}
\setlength{\textwidth}{130mm}
\setlength{\oddsidemargin}{0.5\paperwidth}\addtolength{\oddsidemargin}{-25.4mm}\addtolength{\oddsidemargin}{-0.5\textwidth}
\pagestyle{myheadings}
\markboth{foo}{\sffamily Hogg / A tractable plan to recalibrate XP}
\renewcommand{\newblock}{} % this adjusts the bibliography style.
\frenchspacing\sloppy\sloppypar\raggedbottom

% text macros
\renewcommand{\paragraph}[1]{\bigskip\par\noindent\textbf{#1}~---}
\newcommand{\documentname}{\textsl{Note}}
\newcommand{\sectionname}{Section}
\newcommand{\secref}[1]{\sectionname~\ref{#1}}
\newcommand{\foreign}[1]{\textsl{#1}}

% math macros
\newcommand{\set}[1]{\mathscr{#1}}

\title{\bfseries A computationally tractable plan to recalibrate the XP visit spectra}
\author{Hogg}
\date{May 2025}
\begin{document}
\maketitle

\paragraph{Abstract}
We deliver Mission-averaged XP spectra, and also individual-epoch XP spectra, calibrated to a consistent spectral resolution and throughput.
We learn the mean spectral coefficients for every star in a map operation over stars.
We learn the shapes of the basis functions for every across-scan position in a map operation over position--time bins.
These bins can overlap in across-scan position and in elapsed mission time to encourage smoothness of this non-parametric model.
This approach is computationally tractable, because of the decomposition of the optimization into alternating maps.
It relies on the fact that \textsl{Gaia} has many transits per star, and many transits per across-scan location on the detector.

\paragraph{Setup}
We're going to imagine that there are has transits indexed by $n$ (in which an XP spectrum was read for one star).
we will refer to these individual observations as ``transits'' in what follows.
Each of these transits has pixels $i$, such that the fundamental data are pixel values $f_{ni}$.
We're going to assume that these are all correctly read, and background subtracted and so on, such that they are linearly related to the emission from the star.

Stars will be indexed by $m$, such that transit $n$ is of star $m_n$.
In set terms, transit $n$ is an observation of star $m$ if $n\in\set{S}_m$, where $\set{S}_m$ is the set of all transits of star $m$.
In addition, every transit is taken at an across-scan location $\mu_n$ and time $t_n$ and along-scan sub-pixel offset $\Delta_n$.
(HOGG ASKS: Should I use $\mu$ for the AC coordinate? What is $\Delta$ called in the documentation?)
For computational tractability (and we remind you that we have a lot of data), we will split the across-scan positions and times into bins $\ell$---maybe overlapping bins---such that transit $n$ is an observation in position--time bin $\ell$ if $n\in\set{T}_\ell$, where $\set{T}_\ell$ is the set of all observations that fall into position--time bin $\ell$.
There are so many transits that there could be as many across-scan bins as there are detector positions in the across-scan direction, or even more.
We are recommending that these bins be set up to overlap, to encourage results to be somewhat smooth.
More about that when we run a test.

Now we will depart from the CU5 approach in a simple way.
My model is that the basis functions $B_k()$ are going to depend on position--time bin $\ell$.
That is, we will not learn the convolution kernels $a_{ij}$ but instead we will learn the convolved basis functions.
We will break degeneracies by choosing a fiducial bin $\ell_0$ in which we hard-set the basis functions $B_k()$, and we will get smoothness by having surfeit of data.

Briefly, the plan is: We will optimize the coefficients $b_{mk}$ for star $m$ by performing a linear fit to all observations $n\in\set{S}_m$, and then we will optimize the basis functions $B_{k\ell}()$ by performing a linear fit to all observations $n\in\set{T}_\ell$.
The model is
\begin{align}
    y_{ni} &= \sum_{k=1}^K b_{mk}\,B_{k\ell}(i-\Delta_n)+\text{noise} ~;~ \text{when} ~ n\in\set{S}_m\cap\set{T}_\ell ~.
\end{align}
Because we will make our position--time bins $\ell$ tiny, typically only one transit will be in this set intersection, though there may be exceptions.
This model can be optimized with iterated least squares, in which we fit for the $b_{mk}$ in an asynchronous map over stars, and then fit for the basis functions $B_{k\ell}()$ in an asynchronous map over position--time bins $\ell$.
The only additional constraint is that at fiducial position--time bin $\ell=\ell_0$ (maybe the ``best location'' on the detector), the $B_{k\ell}()$ functions will be set to the \textsl{Gaia} Mission fiducial functions $B_{k0}()$.

Now we need a basis in which to describe the $B_{k\ell}()$ functions.
You aren't going to like our ideas here.
Our recommendation is to use the fiducial basis $B_{k0}()$ as the basis for the $B_{k\ell}()$.
That is
\begin{align}
    B_{k\ell}() = \sum_{k'=1}^{K} a_{\ell kk'}\,B_{k'0}() ~,
\end{align}
where the $a_{\ell kk'}$ are coefficients that mix the fiducial functions $B_{k0}()$ into the functions $B_{k\ell}()$ relevant to each position--time bin $\ell$.
In the fiducial bin $\ell_0$, the coefficients are $a_{\ell_0 kk'} = \delta_{kk'}$, where $\delta_{kk'}$ is the Kronecker delta (or identity).

We are going to assume not just that the $y_{ni}$ are correctly extracted and background-subtracted and so on, but also that they have known noise properties.
Even more specifically, we are going to assume that we have uncertainty variance estimates $\sigma^2_{ni}$ that accurately represent the variances of additive, zero-mean, Gaussian noise.
In this case, the objective function for the entire survey is given by
\begin{align}
    \chi_{ni} &= \frac{1}{\sigma_{ni}}\,\left[y_{ni} - \sum_{k=1}^K b_{m_nk}\,B_{k\ell_n}(i-\Delta_n)\right] \\
    \chi^2 &= \sum_n\sum_i \chi_{ni}^2 ~.
\end{align}

The optimization procedure below will deliver mean spectral coefficients $b_{mk}$ for every star $m$.
But of course stars vary; that's almost the whole point of this project.
Can we output useful spectral information at every transit epoch for every star, on a consistent resolution and throughput grid?
Yes, of course!
For every transit $n$, it is possible to do a least-square fit for the $K$ coefficients $b_{nk}$ relevant to that observation, using the best-fit values for the basis functions $B_{k\ell}()$ in the position--time bin $\ell$ corresponding to transit $n$.
(HOGG ASKS: Need I say more?)

HOGG SAY: I need to say that I can't deal with the different gates, or at least not trivially.

\paragraph{Initialization, optimization, and implementation}
Briefly, we are going to initialize the $a_{\ell kk'}$ trivially, and then perform iterated least squares to update the stellar mean spectral coefficients $b_{mk}$ by what we will call the ``b-step'' and the basis-function coefficients $a_{\ell kk'}$ by what we will call the ``a-step''.
The initialization will be $a_{\ell kk'}=\delta_{kk'}$ for all $\ell$.
Then the b-step and a-step are iterated to convergence.

The \emph{b-step} updates the coefficients $b_{mk}$ for each star $m$.
Each star $m$ has been observed $N_m=|\set{S}_m|$ times.
If there are $D$ pixels in each transit observation (such that $1\leq i\leq D$), then there are $N_m\,D$ measurements relevant to the inference for this star.
We construct a $N_m\,D\times K$ design matrix $X_m$, a $N_m\,D$ column data vector $Y_m$, and a diagonal variance tensor $C_m$ with elements 
\begin{align}
    [X_m]_{jk} &= B_{k\ell_n}(i-\Delta_n) \\
    [Y_m]_j &= y_{ni} \\
    [C_m]_{jj} &= \sigma^2_{ni} ~,
\end{align}
where the index $j$ indexes all $(n,i)$ pairs corresponding to all pixels of all transits for star $m$.
Now we fit to update the $b_{mk}$ with the usual mathematics:
\begin{align}
    Q_m &= [X_m^\top\,C^{-1}_m\,X_m]^{-1}\,X_m^\top\,C^{-1}_m\,Y_m \\
    b_{mk} &\leftarrow [Q_m]_{k}
\end{align}
The b-step uses the current best guess of the basis functions $B_{k\ell}()$, which we update in the a-step (below).

The \emph{a-step} updates the functions $B_{k\ell}()$ by updating the mixing coefficients $a_{\ell kk'}$.
Each position--time bin $\ell$ contains $N_\ell=|\set{T}_\ell|$ transit observations.
There are $N_\ell\,D$ measurements relevant to the inference for this bin.
We construct a $N_\ell\,D\times K^2$ design matrix $X_\ell$, a $N_\ell\,D$ column data vector $Y_\ell$, and a diagonal variance tensor $C_\ell$ with elements 
\begin{align}
    [X_\ell]_{js} &= b_{m_nk}\,B_{k'0}(i-\Delta_n) \\
    [Y_\ell]_j &= y_{ni} \\
    [C_\ell]_{jj} &= \sigma^2_{ni} ~,
\end{align}
where the index $j$ indexes all $(n,i)$ pairs corresponding to all pixels of all transits in position--time bin $\ell$,
and the index $s$ indexes all $K^2$ $(k,k')$ pairs.
Now we fit to update the $a_{\ell kk'}$ with the same usual mathematics:
\begin{align}
    Q_\ell &= [X_\ell^\top\,C^{-1}_\ell\,X_\ell]^{-1}\,X_\ell^\top\,C^{-1}_\ell\,Y_\ell \\
    a_{\ell kk'} &\leftarrow [Q_\ell]_{s}
\end{align}
The a-step uses the current best guess of all of the coefficients $b_{mk}()$, which we update in the b-step (above).

Extremely importantly, \emph{we don't perform the a-step at all for the fiducial position--time bin $\ell_0$};
we keep $a_{\ell_0 kk'}=\delta_{kk'}$ always.
This fixes the $B_{k\ell_0}()$ functions to their fiducial shapes at the fiducial position and time.
We very much hope that this rigidity in the fiducial bin is sufficient to break all degeneracies in the model.
We aren't absolutely certain that this constraint will be strong enough.

Some implementation notes:
The b-step is performed asynchronously in a map operation over stars.
The a-step is performed asynchronously in a map operation over position--time bins.
The matrices $C_m$ and $C_\ell$ are never explicitly built, since the right multiplication by their inverses is a trivial column-wise multiply.
The inverse operation \texttt{inv()} is never used; instead we always perform $M^{-1}\,b$ by an operation like \texttt{solve(M, b)} or \texttt{lstsq(M, b)}.

Once everything is converged, and all the mean spectral coefficients $b_{mk}$ are set,
and all of the basis-function mixing coefficents $a_{\ell kk'}$ are set,
we can obtain best-fit individual-transit coefficients $b_{nk}$ for every single individual transit $n$ by
\begin{align}
    [X_n]_{ik} &= B_{k\ell}(i-\Delta_n) \\
    [Y_n]_i &= y_{ni} \\
    [C_n]_{ii} &= \sigma^2_{ni} \\
    Q_n &= [X_n^\top\,C^{-1}_n\,X_n]^{-1}\,X_n^\top\,C^{-1}_n\,Y_n \\
    b_{nk} &\leftarrow [Q_n]_{k} ~.
\end{align}
This can also be done asynchronously, this time over every single transit $n$.
To be extremely clear: The coefficients $b_{mk}$ are the mean (over all transits) spectral coefficients for each star $m$,
while the coefficients $b_{nk}$ are the individual-transit spectral coefficients for each individual transit $n$.
Roughly speaking (but not precisely, because of heteroskedasticity):
\begin{align}
    b_{mk} \approx \frac{1}{|\set{S}_m|}\sum_{n\in\set{S}_m} b_{nk} ~.
\end{align}
This won't be exactly true, but it will be close, and it will be comforting to our users.

\paragraph{Demonstration and results}

\paragraph{Discussion and issues}
HOGG SAY: Binning is sinning, so what should we do differently? And what would that cost us?

HOGG SAY: We gave up on the multiple gates. How could we handle them?

HOGG SAY: We don't address overall (or absolute) calibration, nor do we address the accuracy of the fiducial basis functions.

\paragraph{Acknowledgments}
Thank you to the CU5 XP team at Cambridge University for helping me formulate this.

\end{document}
