\documentclass{article}
\usepackage[utf8]{inputenc}
\usepackage[a4paper]{geometry}
\usepackage{setspace}
\usepackage{amsmath, amsfonts, mathrsfs}

% typesetting issues
\setstretch{1.08}
\addtolength{\topmargin}{-0.50mm}
\addtolength{\textheight}{1.60in}
\setlength{\textwidth}{128mm}
\setlength{\oddsidemargin}{0.5\paperwidth}\addtolength{\oddsidemargin}{-25.4mm}\addtolength{\oddsidemargin}{-0.5\textwidth}
\pagestyle{myheadings}
\markboth{foo}{\sffamily Hogg / What is a measurement?}
\renewcommand{\newblock}{} % this adjusts the bibliography style.
\frenchspacing\sloppy\sloppypar\raggedbottom

% text macros
\renewcommand{\paragraph}[1]{\bigskip\par\noindent\textbf{#1}~---}
\newcommand{\documentname}{\textsl{Note}}
\newcommand{\sectionname}{Section}
\newcommand{\secref}[1]{\sectionname~\ref{#1}}
\newcommand{\foreign}[1]{\textsl{#1}}

% math macros
\newcommand{\set}[1]{\mathscr{#1}}

\title{\bfseries A computationally tractable plan to recalibrate the XP visit spectra}
\author{Hogg}
\date{May 2025}
\begin{document}
\maketitle

\paragraph{Abstract}
We deliver Mission-averaged XP spectra, and also individual-epoch XP spectra, calibrated to a consistent spectral resolution and throughput.
We learn the mean spectral coefficients for every star in a map operation over stars.
We learn the shapes of the basis functions for every across-scan position in a map operation over position--time bins.
These bins can overlap in across-scan position and in elapsed mission time.
This approach is computationally tractable.
It relies on the fact that \textsl{Gaia} has many transits per star, and many transits per across-scan location on the detector.

\paragraph{Setup}
We're going to imagine that there are has transits indexed by $n$ (in which an XP spectrum was read for one star).
we will refer to these individual observations as ``transits'' in what follows.
Each of these transits has pixels $i$, such that the fundamental data are pixel values $f_{ni}$.
We're going to assume that these are all correctly read, and background subtracted and so on, such that they are linearly related to the emission from the star.

Stars will be indexed by $m$, such that transit $n$ is of star $m_n$.
In set terms, transit $n$ is an observation of star $m$ if $n\in\set{S}_m$, where $\set{S}_m$ is the set of all transits of star $m$.
In addition, every transit is taken at an across-scan location $\mu_n$ and time $t_n$ and along-scan sub-pixel offset $\delta_n$.
(HOGG ASKS: Should I use $\mu$ for this coordinate? What is $\delta$ called in the documentation?)
For computational tractability (and we remind you that we have a lot of data), we will split the across-scan positions and times into bins $\ell$---maybe overlapping bins---such that transit $n$ is an observation in position--time bin $\ell$ if $n\in\set{T}_\ell$, where $\set{T}_\ell$ is the set of all observations that fall into position--time bin $\ell$.
There are so many transits that there could be as many across-scan bins as there are detector positions in the across-scan direction, or even more.
We are recommending that these bins be set up to overlap, to encourage results to be somewhat smooth.
More about that when we run a test.

Now we will depart from the CU5 approach in a simple way.
My model is that the basis functions $B_k()$ are going to depend on position--time bin $\ell$.
That is, we will not learn the convolution kernels $a_{ij}$ but instead we will learn the convolved basis functions.
We will break degeneracies by choosing a fiducial bin $\ell_0$ in which we hard-set the basis functions $B_k()$, and we will get smoothness by having surfeit of data.

Briefly, the plan is: We will optimize the coefficients $b_{mk}$ for star $m$ by performing a linear fit to all observations $n\in\set{S}_m$, and then we will optimize the basis functions $B_{k\ell}()$ by performing a linear fit to all observations $n\in\set{T}_\ell$.
The model is
\begin{align}
    y_{ni} &= \sum_{k=1}^K b_{mk}\,B_{k\ell}(i-\delta_n)+\text{noise} ~;~ \text{when} ~ n\in\set{S}_m\cap\set{T}_\ell ~.
\end{align}
Because we will make our position--time bins tiny, typically only one transit will be in this set intersection, though there may be exceptions.
This model can be optimized with iterated least squares, in which we fit for the $b_{mk}$ in an asynchronous map over stars, and then fit for the basis functions $B_{k\ell}()$ in an asynchronous map over position--time bins $\ell$.
The only additional constraint is that at fiducial position--time bin $\ell=\ell_0$ (maybe the ``best location'' on the detector), the $B_{k\ell}()$ functions will be set to the \textsl{Gaia} Mission fiducial values.

Now we need a basis in which to describe the $B_{k\ell}()$ functions.
You aren't going to like my ideas here.
(HOGG SAY: Write all this down.)

HOGG SAY: Noise model and explicit objective function.

This procedure will deliver mean spectral coefficients $b_{mk}$ for every star $m$.
But of course stars vary; that's almost the whole point of this project.
Can we output useful spectral information at every transit epoch for every star, on a consistent resolution and throughput grid?
Yes, of course!
For every transit $n$, it is possible to do a least-square fit for the $K$ coefficients $b_{nk}$ relevant to that observation, using the best-fit values for the basis functions $B_{k\ell}()$ in the position--time bin $\ell$ corresponding to transit $n$.
(HOGG ASKS: Need I say more?)

HOGG SAY: I need to say that I can't deal with the different gates, or at least not trivially.

\paragraph{Acknowledgements}
Thank you to the XP team at Cambridge for helping me formulate this.

\end{document}
